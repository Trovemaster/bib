%\documentclass[prb,preprint,preprintnumbers,amsmath,amssymb]{revtex4}
\documentclass[prb,preprint,12pt,superscriptaddress,floatfix,bibnotes,nofootinbib,unsortedaddress,preprintnumbers,amsmath,amssymb]{revtex4}



\usepackage{graphicx}% Include figure files
\usepackage{dcolumn}% Align table columns on decimal point
\usepackage{epstopdf}
\usepackage{xcolor} %\fcolorbox{yellow}{yellow}{ }

\usepackage{color}

\newcommand{\red}[1]{{\color{red} #1}}
\newcommand{\blue}[1]{{\color{blue} #1}}
\newcommand{\green}[1]{{\color{green} #1}}

\newcommand{\cm}{cm$^{-1}$}

\newcommand{\p}{^\prime}
\newcommand{\pp}{^{\prime\prime}}


%\nofiles

\newcommand{\ibJ}{\mbox{\itshape\bfseries J}}
\newcommand{\ibL}{\mbox{\itshape\bfseries L}}
\newcommand{\ibN}{\mbox{\itshape\bfseries N}}
\newcommand{\ibS}{\mbox{\itshape\bfseries S}}
\newcommand{\citings}[6]{#1, {#2} \textbf{#3}, #4 (#6).}         % JCP-style
\newcommand{\bibref}[7]{\bibitem{#1}\citings{#2}{#3}{#4}{#5}{#6}{#7}}
\newcommand{\eref}[1]{(\ref{#1})}

\newcommand{\ai}{\textit{ab initio}}
\newcommand{\Ai}{\textit{Ab initio}}

\newcommand{\TROVE}{{\sc TROVE}}
\newcommand{\ohhh}{H$_3$O$^{+}$}
\newcommand{\oddd}{D$_3$O$^{+}$}
\newcommand{\ohhd}{H$_2$DO$^{+}$}
\newcommand{\ohdd}{HD$_2$O$^{+}$}

\newcommand{\Cv}[1]{${\mathcal C}_{#1{\rm v}}$}
\newcommand{\Dh}[1]{${\mathcal D}_{#1{\rm h}}$}
\newcommand{\Ch}[1]{${\mathcal C}_{#1{\rm h}}$}
\newcommand{\Dd}[1]{${\mathcal D}_{#1{\rm d}}$}


\newcommand{\2}{$_{2}$}
\newcommand{\3}{$_{3}$}



\def\deg{$^{\circ}$}
\def\o{\phantom{0}}



%=============================================================================

\begin{document}


\title{Radiative cooling of H$_3$O$^{+}$, HD$_2$O$^{+}$, H$_2$DO$^{+}$ and D$_3$O$^{+}$}

%\title{Stability of ro-vibrational states of H$_3$O$^{+}$, HD$_2$O$^{+}$, H$_2$DO$^{+}$ and D$_3$O$^{+}$}

\author{Vladlen V. Melnikov}%
\thanks{Corresponding author.
E-mail: melnikov@phys.tsu.ru} \affiliation{Siberian Institute of Physics \& Technology,
Tomsk State University, Tomsk, 634050 Russia}%
%Authors' institution and/or address\\
%This line break forced with \textbackslash\textbackslash

\author{Sergei N. Yurchenko}%
\affiliation{Department of Physics \& Astronomy, University College London, London WC1E~6BT, UK}

\author{Jonathan Tennyson}
\affiliation{Department of Physics \& Astronomy, University College London, London WC1E~6BT, UK}

\author{Per Jensen}
\affiliation{FB~C~-- Theoretische Chemie, Bergische Universit{\"a}t Wuppertal, 42097 Wuppertal, Germany}



\date{\today}% It is always \today, today,
             %  but any date may be explicitly specified



\begin{abstract}

    A theoretical study of the radiative lifetimes of hydronium ion and its isotopologues is presented. \textit{Ab initio} potential energy and the dipole moment surfaces of the ground electronic states of H$_3$O$^{+}$ have been used to calculate ro-vibrational energy levels, wavefunctions and Einstein coefficients of the systems under consideration. A detailed analysis of the stability of the ro-vibrational states have been carried out and the most long-living states of the hydronium  ions. The estimated radiative lifetimes and cooling functions for low temperatures (approximately below 100~K) are presented.

    %These estimates can be used to assess the cooling properties of hydronium ion for %applications in storage rings \red{CHECK}.

\end{abstract}


\maketitle


%-----------------------------------------------------------------------------
\section{Introduction}

\blue{
2. Proper quantum number designations for the (rotational) energy levels discussed. The
quantum numbers for H3O+ and D3O+ are non standard but probably sufficient to uniquely
identify each level. This probably acceptable if not ideal (it is better to conventional
notation). For H2DO+ and HD2O+ the quantum numbers given are insufficient to characterise
the level in question and this does not work: a full asymmetric top description ie Ka and Kc
are required.
}

Molecules in Universe are found in a wide variety of environments: from diffuse
clouds with very low temperatures to atmospheres of planets, brown dwarfs and
cool stars which are significantly hotter. In order to investigate the evolution
of such complex systems it is essential to have reasonable models of constituent
species and for this purpose the radiative and cooling properties are of great
importance.

Although interstellar molecular clouds are usually characterised as cold, they
are not really fully thermalized. Whether a species attains thermal equilibrium
with the environment depends on the radiative lifetimes of its states and the
rate for collisional excitations to the states: this is normally
characterised by the critical density. In such regions radiative
lifetimes are also important for models of the many species which are observed
to mase. The long lifetimes associated with certain excited states can lead to
population trapping and non-thermal distributions. Such behaviour was observed
for the H$^+_3$ molecule both in space\cite{02GoMcGe.H3+, 05OkGeGo.H3+} and at
the laboratory,\cite{02KrKrLa.H3+,04KrScTe.H3+} in both cases leading to the
unexpected state distributions.

Dissociative recombination of hydronium has also been extensively studied in ion storage rings.
\cite{96AnHeKe.H3O+,00NeKhRo.H3O+,00JeBiSa.H3O+,10BuStMe.H3O+,10NoBuSt.H3O+}
From the lifetimes calculated below, one can expect that H$_3$O$^{+}$ and it isotopomers
will be the subject of population trapping in a similar manner to that observed
for H$_3^+$ in storage rings. Dissociative recombination of hydronium
has been postulated as the possible cause of emissions of from super-excited
water in cometary coma \cite{jt452} and the mechanism for a spontaneous
infrared water laser \cite{01SaKeWa.H3O+}.

Hydronium and its isotopologues play an important role in planetary and
interstellar chemistry.\cite{00JeBiSa.H3O+,01GoCexx.H3O+} These molecular ions
are found to exist abundantly in both diffuse and dense molecular clouds as well
as in comae. Moreover H$_3$O$^{+}$ is a water indicator of water and can be used to
estimate its abundance when the direct detection is
unfeasible.\cite{92PhVaKe.H3O+} Due to such importance these ions have been the
subject of numerous theoretical and experimental studies (see, for example,
Refs.~%
\onlinecite{00JeBiSa.H3O+,01GoCexx.H3O+, 92PhVaKe.H3O+,  73LiDyxx.H3O+,
80FeHaxx.H3O+, 82SpBuxx.H3O+, 83BeGuPf.H3O+, 84DaHaJo.H3O+, 84BuAmSp.H3O+,
85BeSaxx.H3O+, 85LiOkxx.H3O+, 86DaJoHa.H3O+, 87GrPoSa.H3O+, 88HaLiOk.H3O+,
88VeTeMe.H3O+, 90OkYeMy.H3O+, 90PeNeOw.H3O+,
91HoPuOk.H3O+, 97UyWhOk.H3O+, 99ArOzSa.H3O+, 00ChJuGe.H3O+, 01AiOhIn.H3O+,
02ErSoDo.H3O+, 02CaWaZu.H3O+, 03RaMiHa.NH3, 03HuCaBo.H3O+, 05YuBuJe.H3O+,
06DoNexx.H3O+, 08FuFaxx.H3O+, 09YuDrPe.H3O+, 10MuDoNe.H3O+,
12PeWeBe.H3O+,10BuStMe.H3O+} and references therein) mainly devoted to the
spectroscopy and chemistry of the species. Cooling function of the H$_3^{+}$ ion
has been extensively studied by Miller et al.\cite{jt181,10MiStMe.H3+,13MiStTe.H3+}

Radiative and cooling properties for H$_3$O$^{+}$, HD$_2$O$^{+}$, H$_2$DO$^{+}$
and D$_3$O$^{+}$ are not available in the literature. In the present
work a theoretical study of the ro-vibrational states of these ions is carried
out.  Using  an \ai\  potential energy surface (PES) and an \ai\ dipole moment surface
(DMS) \cite{15OwYuPo.H3O+} for the ground electronic states of H$_3$O$^{+}$,
ro-vibrational energy levels, wavefunctions and Einstein coefficients of the
systems under consideration are computed  employing the nuclear-motion program TROVE.\cite{TROVE,15YaYuxx.method}
Lifetimes of individual states and overall cooling rates are calculated and analyzed.
Recently the same methodology was used to estimate the sensitivities of
hydronium ions  transitions to the variation of the proton-to-electron mass
ratio.\cite{15OwYuPo.H3O+}

We present a detailed analysis of the stability of the ro-vibrational states and
identify the most long-living states of these ions. This study is based on the
methodology \cite{16TeHuNa.method} developed very recently as part of the ExoMol
project \cite{12TeYuxx.db}. The ExoMol projects aims at the comprehensive
description of spectroscopic properties of molecules important for atmospheres
of exoplanets and cool stars. The molecular lifetimes and cooling functions are
now part of the new ExoMol data format \cite{16TeYuAl.db}.



%-----------------------------------------------------------------------------
\section{Theory and computation}

Ro-vibrational energy levels and wavefunctions of the ions under study were
calculated variationally employing the \TROVE\ program.
\cite{TROVE,15YaYuxx.method} Similar approach has already been successfully used
for ro-vibrational calculations of the XY\3\ type molecules
\cite{08YuThCa.NH3+,10YuCaYa.SbH3,13SoYuTe.PH3,13UnTeYu.SO3,14UnYuTe.SO3,
15SoAlTe.PH3,15AdYaYuJe.CH3} including ammonia, \cite{09YuBaYa.NH3,11YuBaTe.NH3}
 which is also characterized by a non-rigid `umbrella' motion as \ohhh.
 The
inversion barrier of \ohhh\ is 1167~\cm, which is lower than 1791~\cm\ found
for of ammonia.\cite{03RaMiHa.NH3} As a result the inversion splitting is
significantly larger, 55.35~\cm\ (Ref.~\onlinecite{99TaOkxx.H3O+}) against
0.793~\cm\ (Ref.~\onlinecite{03RaMiHa.NH3}).

We used the \ai\ PES and DMS of \ohhh\ \cite{15OwYuPo.H3O+,15Yuxxxx.15NH3}
computed using the MRCI/aug-cc-pwCV5Z (5Z) and MRCI/aug-cc-pwCVQZ (QZ) levels of
theory. The complete basis set (CBS) extrapolation was used for the
Born-Oppenheimer PES (see Ref.~\onlinecite{15OwYuPo.H3O+} for details).


For all \TROVE\ ro-vibrational computations the orders of kinetic and
potential energy expansions were set to 6 and 8, respectively. We used the
Morse-type basis functions for stretching modes and numerical basis functions
(numerical solutions of corresponding 1D problem obtained within the framework
of the Numerov-Cooley scheme) for bending vibrations. The \ai\ equilibrium
\ohhh\ structure is characterized by the O-H bond length of about 0.9758~{\AA}
and H-O-H angle close to 111.95\deg. The vibrational basis set is controlled by
the polyad number defined by
\begin{equation}
\label{e:polyad}
P = 2(v_1 + v_2 + v_3) + v_4 + v_5 + v_6/2,
\end{equation}
where $v_1$, $v_2$, $v_3$ represent the quanta of the stretching, $v_4$ and
$v_5$ are of the asymmetric bending and $v_6$ is of the inversion primitive
basis set functions. For this work the maximal polyad  number $P_{\rm max}$ was
set to 28. The computational details of the basis set construction can be found
in Ref.~\onlinecite{09YuBaYa.NH3} as well as the details of the calculations of
Einstein coefficients $A_{if}$. The latter were computed for all possible
initial, $i$, and final, $f$, states lying below 6000~cm$^{-1}$ relatively to
the zero point energy. According to our estimations this energy range is
sufficient to deal with thermodynamic temperatures up to 100~K.




The lifetimes of the states were computed as given by\cite{16TeHuNa.method}
\begin{equation}
  \tau_i = \frac{1}{\sum_f A_{if}},
\end{equation}
where the summation is taken over all possible the \textit{final} states $f$
states for the given \textit{initial} state $i$. The Einstein coefficients (1/s)
are defined as follows:
\begin{equation}
A_{if} = \frac{8\pi^{4}\tilde{\nu}_{if}^{3}}{3h}(2J_{i} + 1)\sum_{A=X,Y,Z}
|\langle \Psi^{f} | \bar{\mu}_{A} | \Psi^{i} \rangle |^{2},
\label{e:A}
\end{equation}
where  $h$ is Planck's constant, $\tilde{\nu}_{if}$ (\cm) is the wavenumber of
the line, (\(hc \,
\tilde{\nu}_{if} = E_{f} -E_{i}\)), $J_{i}$ is the rotational quantum number for
the initial state, \(\Psi^{f}\) and \(\Psi^{i}\) represent the ro-vibrational
eigenfunctions of the final and initial states respectively, \(\bar{\mu}_{A}\)
is the electronically averaged component of the dipole moment (Debye) along the
space-fixed
axis \(A=X,Y,Z\) (see also \citet{05YuThCa.method}).

At the temperature $T$ the cooling function, $W(T)$, is the total power emitted
by a molecule and is given by the following expression:\cite{16TeYuAl.db}
\begin{equation}
\label{e:cooling}
  W(T) = \frac{1}{4\pi Q(T)} \sum_{i,f} A_{if} h c \tilde{\nu}_{if} (2 J_i+1)
g_i \exp\left( \frac{-c_2 \tilde{E}_i}{T} \right) ,
\end{equation}
where $\tilde{\nu}_{if}$ is frequency wavenumber of the transition $i \to f$,
$J_i$ is the rotational quantum number of the initial state and $g_i$ is its
nuclear spin degeneracy factor. The exponential term is the Boltzmann factor,
$c_2$ is the second radiation constant and $\tilde{E}_i$ is the wavenumber of
the corresponding state. The partition function, $Q(T)$, is defined as
\begin{equation}
  Q(T) = \sum_{i} g_i  (2J_i+1) \exp\left( \frac{-c_2 \tilde{E}_i}{T} \right) .
\end{equation}
Partition functions were computed for each ion employing complete sets of
obtained ro-vibrational energies.

The same PES and DMS were used for each isotopologue, meaning that
no allowance was made for the failure of the Born-Oppenheimer approximation.
The energies of all four
isotopologues of \ohhh\ considered are very different not only due to the mass
changes, but also due to the different symmetries these species belong to. We
used molecular symmetry group  \Dh{3}(M) to classify the ro-vibrational
states of the symmetric species \ohhh\ and \oddd, and \Cv{2}(M) for the
asymmetric istopologues \ohhd\ and \ohdd. The difference in the Einstein
coefficients are also quite profound, especially between symmetric and
asymmetric isotopologues. Special attention should be paid to the situation when
isotope substitution of an ion is considered. In this case the position of the
center-of-mass is displaced and, as a consequence, the total molecular dipole
moment is changed due to its nuclear contribution (see, for example,
Ref.~\onlinecite{88Jensen.CH2}). The lifetimes of different isotopologues are
also expected to differ significantly due to the factors above as well as due to
the different selection rules and nuclear statistics.
Selection rules for $J$ are
\begin{equation}
 J\p - J\pp = 0,\pm 1  \quad {\rm and} \quad  J\p - J\pp > 0.
\end{equation}
The symmetry selection rules for \ohhh\ are
\begin{equation}
 A_2\p \leftrightarrow A_2\pp , \quad
  E\p \leftrightarrow E\pp,
\end{equation}
for \oddd\ are
\begin{equation}
 A_1\p \leftrightarrow A_1\pp , \quad  A_2\p \leftrightarrow A_2\pp , \quad
  E\p \leftrightarrow E'',
\end{equation}
while those for \ohhd\ and \ohdd\ are
\begin{equation}
 A_1 \leftrightarrow A_2, \quad B_1 \leftrightarrow B_2.
\end{equation}

%n decreasing order of the symmetry degeneracy gns: ortho (E, gns = 16), meta (A1, gns = 10) and para (A2, gns = 1).

Nuclear spin statistics\cite{98BuJexx.method} results in three distinct forms of \oddd, so-called ortho (A$_1$, $g_{ns}=10$), meta (E, $g_{ns}=8$) and para (A$_2$, $g_{ns}=1$). For \ohhh, only ortho (A$_2$, $g_{ns}=4$) and para (E, $g_{ns}=2$) states exist, and ro-vibrational states with symmetries of A$_1'$ and A$_1''$ are missing. \ohhd\ and \ohdd\ both support ortho and para states: B$_1$ and B$_2$ are ortho ($g_{ns}=9$), and A$_1$ and A$_2$ are para ($g_{ns}=3$) for \ohhd, with ortho and para labels swapped for \ohdd, i.e. A$_1$ and A$_2$ are ortho ($g_{ns}=12$), and B$_1$ and B$_2$ are para ($g_{ns}=6$).


%Since Eqs.~(\ref{eq:A21}--\ref{eq:partition}) depend on energy eigenvalues, corresponding wavefunctions, electric dipole moment %and symmetries along with nuclear spin statistics. All these properties are affected when constituent atoms are substituted by %their isotopes.


%-----------------------------------------------------------------------------
\section{Results and discussion}


The quality of the spectroscopic model employed here for the lifetime
calculations along with the \ai\ PES and variational method  is demonstrated by
Table~\ref{tab:Expt}, which compares calculated energies of the isotopologues of
\ohhh\ with available experimental values. These are very good results for a
pure \ai\ PES and suggesting that our model should be more than appropriate for purposes of the present
study.

%especially considering that matrix elements of the dipole moment are not very sensitive to the quality of the underlying PES.


\begin{table}[!htbp]
  \tabcolsep=0.5cm
  \caption{\label{tab:Expt} Vibrational energies in cm$^{-1}$: comparison with experiment.}
  \renewcommand{\arraystretch}{0.75}
  \begin{tabular}{lcdcdd}
  \hline\hline
  State & Sym. & \multicolumn{1}{c}{Expt.} & \multicolumn{1}{c}{Ref.} & \multicolumn{1}{c}{Calc.} & \multicolumn{1}{c}{Diff.} \\ \hline

  \multicolumn{6}{l}{H$_3$O$^+$} \\

  $\nu_2^+$   & $A_1$ &  581.17& [\onlinecite{86LiOkSe.H3O+}] &  579.07 &  2.10 \\
  $2\nu_2^+$  & $A_1$ & 1475.84& [\onlinecite{86DaJoHa.H3O+}] & 1470.67 &  5.17 \\
  $\nu_1^+$   & $A_1$ & 3445.01& [\onlinecite{99TaOkxx.H3O+}] & 3442.61 &  2.40 \\
  $\nu_3^+$   &  $E$  & 3536.04& [\onlinecite{99TaOkxx.H3O+}] & 3532.58 &  3.46 \\
  $\nu_4^+$   &  $E$  & 1625.95& [\onlinecite{87GrPoSa.H3O+}] & 1623.02 &  2.93 \\
  $0^-$       & $A_1$ &   55.35& [\onlinecite{99TaOkxx.H3O+}] &   55.03 &  0.32 \\
  $\nu_2^-$   & $A_1$ &  954.40& [\onlinecite{86LiOkSe.H3O+}] &  950.94 &  3.46 \\
  $\nu_1^-$   & $A_1$ & 3491.17& [\onlinecite{99TaOkxx.H3O+}] & 3488.32 &  2.85 \\
  $\nu_3^-$   &  $E$  & 3574.29& [\onlinecite{99TaOkxx.H3O+}] & 3571.04 &  3.25 \\
  $\nu_4^-$   &  $E$  & 1693.87& [\onlinecite{87GrPoSa.H3O+}] & 1690.65 &  3.22 \\

  \multicolumn{6}{l}{D$_3$O$^+$} \\

  $\nu_2^+$   & $A_1$ &  453.74& [\onlinecite{90PeNeOw.H3O+}]&   451.58 &  2.16 \\
  $\nu_3^+$   & $ E $ & 2629.65& [\onlinecite{90PeNeOw.H3O+}]&  2627.14 &  2.51 \\
  $0^-$       & $A_1$ &   15.36& [\onlinecite{98ArOzSa.H3O+}]&    15.38 & -0.02 \\
  $\nu_2^-$   & $A_1$ &  645.13& [\onlinecite{90PeNeOw.H3O+}]&   642.79 &  2.34 \\
  $\nu_3^-$   & $ E $ & 2639.59& [\onlinecite{90PeNeOw.H3O+}]&  2637.10 &  2.49 \\



  \multicolumn{6}{l}{H$_2$DO$^+$} \\

  $0^-$       & $B_1$ &     40.56& [\onlinecite{08FuFaxx.H3O+}]&      40.39 &   0.17 \\
  $\nu_1^+$   & $A_1$ &   3475.97& [\onlinecite{06DoNexx.H3O+}]&    3473.27 &   2.70 \\
  $\nu_1^-$   & $B_1$ &   3508.63& [\onlinecite{06DoNexx.H3O+}]&    3505.51 &   3.12 \\
  $\nu_3^+$   & $B_2$ &   3531.50& [\onlinecite{06DoNexx.H3O+}]&    3528.07 &   3.43 \\
  $\nu_3^-$   & $A_2$ &   3556.94& [\onlinecite{06DoNexx.H3O+}]&    3553.63 &   3.31 \\

  \multicolumn{6}{l}{HD$_2$O$^+$} \\

  $0^-$       & $B_1$ &   26.98& [\onlinecite{08FuFaxx.H3O+}]&    26.92 &  0.06 \\


  \hline\hline

  \end{tabular}
%  \parbox{10.7cm}
  %{\scriptsize
  %\begin{flushleft}
  %$^{\rm a}$ Ref.~\onlinecite{86LiOkSe.H3O+}. \\
  %$^{\rm b}$ Ref.~\onlinecite{86DaJoHa.H3O+}. \\
  %$^{\rm c}$ Ref.~\onlinecite{99TaOkxx.H3O+}. \\
  %$^{\rm d}$ Ref.~\onlinecite{87GrPoSa.H3O+}. \\
  %$^{\rm e}$ Ref.~\onlinecite{90PeNeOw.H3O+}. \\
  %$^{\rm f}$ Ref.~\onlinecite{98ArOzSa.H3O+}.
  %\end{flushleft}}

\end{table}


An overview of the lifetimes of the ro-vibrational states ($J<6$) for the
hydronium ions under study is given in Fig.~\ref{fig:lifetime-1}.  In general
the lifetimes exhibit the expected gradual reduction with the increase of energy
with some islands of meta-stable states.  The complete list of lifetimes for all
four isotopologues are give as supplementary material to this paper.

%\red{CAN WE PUT THE FOLLLOWING INTO A TABLE?}
%The most long-living states identified are the following.
%For H$_3$O$^+$ there are
%$(0,0,A_2'')$, $\tau=23.9$ years;
%$(1,1,E')$, $\tau=26.2$ years;
%$(3,3,A_1')$, $\tau=17.7$ years;
%$(4,1,E'')$, $\tau=140$ years.
%For D$_3$O$^+$ there are
%$(1,1,E')$, $\tau=857$ years;
%$(2,1,E')$, $\tau=190.4$ years;
%$(3,1,E')$, $\tau=594.4$ years;
%$(4,1,E'')$, $\tau=3816$ years.
%For H$_2$DO$^+$ there are
%$(1,1,B_2)$, $\tau=3$ days;
%$(2,0,A_1)$, $\tau=265.4$ days;
%$(2,2,A_1)$, $\tau=89.1$ days;
%$(1,1,A_2)$, $\tau=4.8$ days.
%And for HD$_2$O$^+$ there are
%$(1,1,B_2)$, $\tau=1.8$ days;
%$(0,0,B_1)$, $\tau=21.6$ days;
%$(2,1,B_2)$, $\tau=1.4$ days;
%$(1,1,A_1)$, $\tau=8.9$ days.

\begin{table}[!htbp]
  \tabcolsep=0.5cm
  \caption{\label{tab:LF} Longest lifetimes of four ion hydronium ions: all states considered
are for the vibration-inversion ground state. The notations $(J,K,\Gamma,{\rm state})$ and $(J_{K_a,K_c},\Gamma,{\rm state})$ are used for symmetric and asymmetric isotopolouges are used.}
  \renewcommand{\arraystretch}{0.75}
  \begin{tabular}{ldd}
  \hline\hline
  State & \multicolumn{1}{c}{Energy, cm$^{-1}$} & \multicolumn{1}{c}{$\tau$} \\ \hline
\multicolumn{1}{l}{\ohhh}  & & \multicolumn{1}{c}{(years)}         \\
$(1,0,A_2\p,0^+)$         &   22.47 &             \infty    \\
$(1,1,E\pp,0^+)$          &   17.38 &             \infty    \\
$(3,3,A_2\pp,0^+)$        &   88.96 &             \infty    \\
 $(5,5,E\pp,0^+)$         &  209.58 &            140    \\
 $(2,1,E\pp,0^+)$         &   62.29 &            26.2   \\
 $(2,2,E\p,0^+)$          &   47.03 &            23.9   \\
 $(4,4,E\p,0^+)$          &  144.13 &            17.7   \\
\hline
\multicolumn{1}{l}{\oddd}  & & \multicolumn{1}{c}{(years)}        \\
$(0,0,A_1\p,0^+)$         &  0.00 &             \infty    \\
$(1,0,A_2\p,0^+)$         &  11.33 &             \infty     \\
$(1,1,E\pp,0^+)$          &  8.78 &             \infty    \\
$(3,3,A_2\pp,0^+)$        &  45.02 &             \infty    \\
$(5,5,E\pp,0^+)$          & 106.15 &         3816   \\
$(2,2,E\p,0^+)$           &  23.79 &          857  \\
$(4,4,E\p,0^+)$           &  72.48 &          594.4   \\
$(3,3,A_1\pp,0^+)$           &  45.02 &          190.4   \\
\hline
\multicolumn{1}{l}{\ohhd} & &  \multicolumn{1}{c}{(days)}         \\
$(0_{0,0}^+,A_1)$           &  0.00 &             \infty    \\
$(1_{1,1}^+,B_1)$           &  15.70 &             \infty    \\
$(1_{1,0}^+,B_2)$           &  18.07 &          265.4   \\
$(2_{2,1}^+,A_2)$           &  55.82 &           89.1   \\
$(2_{2,0}^+,A_1)$           &  56.60 &            4.8   \\
$(1_{0,1}^+,A_2)$           &  11.69 &            3   \\

\hline
\multicolumn{1}{l}{\ohdd} & &  \multicolumn{1}{c}{(days)}         \\
$(0_{0,0}^{+},A_1)$           &  0.00  &             \infty    \\
$(1_{1,1}^{+},B_1)$           &  9.53  &             \infty    \\
$(1_{1,0}^{+},B_2)$           &  14.24 &           21.6   \\
$(2_{2,1}^{+},A_2)$           &  35.35 &            8.9   \\
$(1_{0,1}^{+},A_2)$           &  12.19 &            1.8   \\
$(2_{2,0}^{+},A_1)$           &  27.77 &            1.4   \\
  \hline\hline
  \end{tabular}

\end{table}


Lifetimes $\tau$  of the most long-lived states of the ions are
presented  in Table~\ref{tab:LF}, which
lists states characterized by the longest lifetimes.
Infinitely long lifetimes can be expected for the lowest state of each nuclear-spin symmetry
corresponding ortho and para (and meta for \oddd). In addition the $(1_{1,1}^+,B_1)$ state
for both \ohhh\ and \oddd\ cannot decay via any dipole-allowed transitions so shows
up as having an infinite lifetime.

The meta-stable states with the longest lifetimes correspond to the low-lying,
pure rotational (small $J$) states, which have the smallest numbers of channels
and/or the lowest probability for transitions downwards. Therefore the meta-stable
states of the more symmetric species \ohhh\ and \oddd\ (\Dh{3}) with stricter
selection rules live significantly longer (tens to thousands of years) compared
to the days of \ohdd\ and \ohhd\ (\Cv{2}). For example, \oddd\ has four meta-states
with the lifetimes longer than 100 years. The longest lived of
these, at 3816 years,  is the state $(J=5,K=5,E\pp,0^+)$. In comparison, the longest-lived meta-state
of \ohhd\ $(1_{1,0}^+,B_2)$ has a lifetime of 265 days.

Another important effect from the symmetry lowering \Dh{3} $ \to$  \Cv{2}\ is
related to the non-zero perpendicular dipole moment components of \ohhd\ and
\ohdd. This is also illustrated in Fig.~\ref{fig:CM}. The center-of-mass of an
asymmetric ion is shifted from its geometric center, which causes a non-zero $x$
dipole moment component. Therefore the orthogonal transitions ($\Delta K=\pm 1$)
of \ohhd\ and \ohdd\ should be stronger than those of \ohhh\ and \oddd.
Besides, this component is bigger for HD$_2$O$^+$ than for H$_2$DO$^+$ owing to
the greater displacement of the molecule center-of-mass. This is the likely  reason
why lifetimes of the HD$_2$O$^+$ states are shorter on average compared to those of
H$_2$DO$^+$. The $z$ dipole moment component also changes with isotopic
substitutions, see Fig.~\ref{fig:CM}.

The lifetimes for the longest living states from Table~\ref{tab:LF}
of \oddd\ are about 25 times longer than of \ohhh. Apparently,
this is mainly caused by the shrinking of separation between the ro-vibrational
energy levels (and lowering of $\tilde{\nu}_{if}$ in Eq.~(\ref{e:A})) which
results in lower Einstein-A coefficients and thus higher $\tau$.

It is interesting to note the presence of vibrationally excited meta-stable
states in Fig.~\ref{fig:lifetime-1} between 1000 and 2000~\cm. For example, the
($J=4, K=4, E\pp$, $\nu_4^-$)  state of \ohhh\ has the significant lifetime of
22~s. Similarly, the state ($J=4, K=4, A_1\pp$, $\nu_4^-$ ) of \oddd\ lives 39~s.


%The transitions to the H$_2$DO$^+$ and HD$_2$O$^+$ forms go with the $D_{3h} \to C_{2v}$ symmetry lowering, which leads to the significant reduction of lifetimes. A certain role here also plays dipole moment transformation, since for partly deuterated isotopologues the $x$-component of dipole moment vector becomes nonzero.




\begin{figure}[!htbp]
\begin{tabular}{cc}
  \includegraphics[width=8cm]{figs/h3op-1c.eps} & \includegraphics[width=8cm]{figs/d3op-1c.eps} \\

  \includegraphics[width=8cm]{figs/h2dop-1c.eps}& \includegraphics[width=8cm]{figs/hd2op-1c.eps}\\

\end{tabular}

  \caption{\label{fig:lifetime-1} Calculated lifetimes of the ro-vibrational states ($J<6$) of H$_3$O$^+$ and isotopologues. Lifetime values are plotted in logarithmic scale.}

\end{figure}



Calculated radiative cooling functions of H$_3$O$^+$ and isotopologues are shown
in Fig.~\ref{fig:cooling}. One can see that at  temperatures above 30~K
the cooling $W(T)$  decreases with increasing numbers of deuterium
atoms. This can be easily understood from Eqs.~(\ref{e:A},\ref{e:cooling}):
$W(T)$ is proportion to $\tilde\nu_{if}^4$, and $\tilde{\nu}_{if}$ is
approximately inverse proportional to the mass of hydrogen for the rotational
states (populated at these temperatures).


Therefore at moderate and high temperatures the lighter isotopologues are better
coolers. However, Fig.~\ref{fig:cooling} shows that at lower temperatures their
roles are changed: \ohhh\ is an inefficient cooler and cooling by the
heavier species begins to dominate. This is due to (i) the larger
rotational constants and thus larger separations between the lowest (coldest)
meta-stable rotational states of \ohhh, \ohhd, \ohdd, and \oddd\, which are
47.03~\cm\ ($2,2,E\pp$, $0^-$ \red{CHECK AND ADD THE INVERSION STATE}), 11.70~\cm\ (1,0,$A_2$), 12.19~\cm\ (1,0,$A_2$), and
23.8~\cm\ (2,2,$E\p$), respectively and (ii) fewer states of \ohhh\ due to the
nuclear statistics. Therefore it is more difficult to cool \ohhh\ than \oddd\ at
temperatures below 20--50~K.

The longest meta-stable states (in the absence of the collisions and stimulated
emissions) are found for \oddd: the population in the ($J=4,E\pp$) state can be
trapped for 3816 years. This is relatively `hot' state (160~K), at least for the
standards of the molecular cooling, e.g. storage rings. Such meta-stable states
will hamper cooling of this molecule to the few Kelvin level. The molecule with
the shortest-lived meta-stable states is \ohdd, which can still live for days.


%\red{DOES NOT IT CONTRADICT THE COOLING FUNCTION?}




%\begin{figure}[!htbp]
%\begin{tabular}{cc}
%  \includegraphics[width=8cm]{figs/h3op-2c.eps} & \includegraphics[width=8cm]{figs/d3op-2c.eps} \\
%  \includegraphics[width=8cm]{figs/h2dop-2c.eps}& \includegraphics[width=8cm]{figs/hd2op-2c.eps}\\
%\end{tabular}
%  \caption{\label{fig:lifetime-2} Calculated lifetimes of the most long-living ro-vibrational states of H$_3$O$^+$ and isotopologues.  Energy states are labeled as $(J,K,\Gamma)$. \red{Vlad, please increase the font of axes title, units, labels (i.e. everything)  by a factor of 2}}
%\end{figure}


\begin{figure}%[!htbp]

  \includegraphics[width=9cm]{figs/cooling-log.eps}

  \caption{\label{fig:cooling} Calculated cooling functions of H$_3$O$^+$ and isotopologues.}


\end{figure}


\begin{figure}%[!htbp]

  \includegraphics[width=9cm]{figs/CM_change.eps}

  \caption{\label{fig:CM} Illustration of the change of the centers-of-mass due to the isotopic substitutions in \ohhh.}

\end{figure}



%-----------------------------------------------------------------------------
\section{Conclusions}


A theoretical study of the ro-vibrational states of four hydronium ion
isotopologues has been carried out. \textit{Ab initio} potential energy and
electric dipole moment surfaces were used to calculate ro-vibrational energy
levels, corresponding wavefunctions and Einstein coefficients for these ions.
The results of an analysis of the stability of the ro-vibrational states,
calculated radiative lifetimes and cooling functions for temperatures below
100~K are presented.

Our calculations show that the isotopic substitution with deuterium and
especially the associated asymmetry of the centre-of-mass has a significantly
effect on the lifetimes. A number of long-living meta-stable states are
identified, which can lead to corresponding population trapping. The cooling
function of the main (lightest) isotopologue dominates at higher temperatures
($T>30$~K) among the four spices studied. However this changes it the very low
temperatures, when the cooling of \ohhh\ becomes significantly less efficient
due to the large separation of the energies and also absence of the $A_1\p$ and
$A_2\pp$ states.

The results obtained can be used to assess the cooling properties of hydronium ion,
for applications in ion storage rings and elsewhere.

%Our calculations will be useful for estimating the cooling properties of hydronium ions, for %example, storage rings, ro-vibrational relaxation in the stored ion beam.


%-----------------------------------------------------------------------------
\begin{acknowledgments}

This work was supported in part by ``The Tomsk State University Academic D.I. Mendeleev Fund Program'' grant No. 8.1.51.2015, and in part by the ERC under the Advanced Investigator Project 267219 (SNY). SNY and JT thank the support of the COST action MOLIM (CM1405). We thank Andreas Wolf for suggesting the idea of this work.



\end{acknowledgments}


%\begin{thebibliography}{99}
%\baselineskip=24pt

%\input{paper16-1-refs.tex}


%\end{thebibliography}

%\bibliographystyle{elsarticle-num}
\bibliography{journals_phys,H3O+,SbH3,H3+,NH3,15NH3,linelists,methods,programs,PH3,SO3,NH3p,CH2,CH3,jtj}




\end{document}


