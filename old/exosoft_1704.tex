\documentclass[12pt]{article}

\setlength{\oddsidemargin}{0in}  %left margin position, reference is one inch
\setlength{\textwidth}{6.5in}    %width of text=8.5-1in-1in for margin
\setlength{\topmargin}{-0.5in}    %reference is at 1.5in, -.5in gives a start of about 1in from top
\setlength{\textheight}{9in}     %length of text=11in-1in-1in (top and bot. marg.)


\usepackage{amsmath,amssymb}
\usepackage{graphicx}% Include figure files
\usepackage{caption}
\usepackage{color}% Include colors for document elements
\usepackage{dcolumn}% Align table columns on decimal point
\usepackage{bm}% bold math
\usepackage[numbers,super,comma,sort&compress]{natbib}
%\usepackage[nolists, nomarkers, figuresfirst]{endfloat}

\newcommand{\red}[1]{{\color{red} #1}}
\newcommand{\blue}[1]{{\color{blue} #1}}
\newcommand{\green}[1]{{\color{green} #1}}
\newcommand{\cyan}[1]{{\color{cyan} #1}}
\newcommand{\magenta}[1]{{\color{magenta} #1}}
\newcommand{\yellow}[1]{{\color{yellow} #1}}

\newcommand{\trove}{{\sc TROVE}}
\newcommand{\schr}{Schr\"{o}dinger}
\newcommand{\ai}{\textit{ab initio}}
\newcommand{\Ai}{\textit{Ab initio}}

\newcommand{\p}{^\prime}
\newcommand{\pp}{^{\prime\prime}}
\newcommand{\ppp}{^{\prime\prime\prime}}
\newcommand{\ket}[1]{\vert #1 \rangle  }
\newcommand{\bra}[1]{\langle #1 \vert  }


\newcommand{\Cv}[1]{${\mathcal C}_{#1{\rm v}}$}
\newcommand{\Dh}[1]{${\mathcal D}_{#1{\rm h}}$}
\newcommand{\Ch}[1]{${\mathcal C}_{#1{\rm h}}$}
\newcommand{\Dd}[1]{${\mathcal D}_{#1{\rm d}}$}
\newcommand{\Td}{${\mathcal T}_{\rm d}$}
\newcommand{\Oh}{${\mathcal O}_{\rm h}$}
\newcommand{\Cs}{${\mathcal C}_{\rm s}$}


\newcommand{\2}{$_{2}$}
\newcommand{\3}{$_{3}$}
\newcommand{\4}{$_{4}$}


\definecolor{background-color}{gray}{0.98}

\title{The ExoMol project: Software for computing molecular line lists}
\author {Jonathan Tennyson and  S.N. Yurchenko\\
Department of Physics and Astronomy, University College London,\\ London, WC1E 6BT, UK}
\begin{document}

\maketitle

\begin{abstract}

duo

1. Lande g factors
2. Resonances
3. wavefn plotting


dvr3d

0. Resonances
1.Various algorithmic updates
2. future vibronic




trove
1 curvilinear
2. GAIN
3.euler symmetry factorisation
4. linear TROVE


mention angmol

support software
exocross

\end{abstract}

\section {Introduction}

The ExoMol projects aims to compute line lists of molecular transitions
which are important for the study of hot atmospheres, particularly those
of exoplanets, brown dwarfs and cool stars.\cite{jt528} In practice
these line lists are also useful for a variety of terrestrial applications
as well as for models of non-thermal environments such as masers.
The project has produced comprehensive line lists for a number of molecules
including
BeH,
MgH and
CaH \cite{jt529},
SiO \cite{jt563},
HCN/HNC \cite{jt570},
CH$_4$ \cite{jt564},
NaCl and
KCl \cite{jt583},
PN \cite{jt590},
PH$_3$ \cite{jt593},
H$_2$CO \cite{jt593},
AlO \cite{jt598},
NaH \cite{jt605},
HNO$_3$ \cite{jt614},
CS \cite{jt615},
CaO \cite{jt618},
SO$_2$ \cite{jt635},
HOOH \cite{jtHOOH},
H$_2$S \cite{jth2s},
VO \cite{jtVO}
SO$_3$ \cite{jtSO3}
and CrH \cite{jtCrH};
the diatomic studies generally include consideration
of all important isotopologues. These line lists are large with,
for example, the line list for the diatomic $^{40}$Ca$^{16}$O contains
over 28 million transitions \cite{jt618}, and the line lists
for the polyatomic systems CH$_4$, PH$_3$, H$_2$CO, HOOH and SO$_3$
with 10 billion or more lines.

Computing these line lists has led us to develop or improve specialist
programs designed to study the nuclear motion problem of the various
molecules under consideration.  This software update describes these
developments.  A common theme of all these programs is the direct
solution of the nuclear motion Schr\"odinger equation using a
variational treatment.  The next section we outline the overall
methodological approach adopted by ExoMol. In the following sections
we consider the main programs used under the project. They are grouped
by the type of system studied. Section 3 considers diatomic systems,
for which we use Le Roy's program {\sc Level} \cite{lr07} and our
program especially developed for the project, {\sc Duo}.\cite{jt609}
Unlike the other programs considered here, {\sc Duo} is designed for
the calculation of vibronic spectra and can treat problems involving
coupled potential energy curves. Section 4 considers triatomic systems
for which the exact kinetic energy nuclear motion code {\sc DVR3D}
\cite{DVR3D} has been employed. For tetratomic systems calculations
have largely been performed with {\sc TROVE} \cite{07YuThJe.method}
although {\sc WAVR4} \cite{WAVR4} has also been tested.\cite{t553}
{\sc TROVE}, which has also been used to study methane, will be
considered in section 5.  Finally a new hybrid methodology based on
the combined use of the variational principle and perturbation theory
has been developed in larger systems. This will be considered in
section 6. The final section considers our conclusions and prospects
for the future,

A number of other groups are involved in projects computing extensive
molecular line lists for astronomical or other purposes, again largely
using especially developed software for the nuclear motion problem.
These include the NASA Ames group of Huang, Schwenke and Lee who have
used the polyatomic nuclear motion program {\sc VTET}
\cite{VTET} and have also been undertaking theoretical
developments.\cite{15Schwenke.diatom}  Tyuterev and Rey from the 
University of Rheims in collaboration with Nikitin from the Tomsk Institute of Atmospheric Optics  have computed line lists for a number of polyatomic species using either their variational polyatomic code
or contact transformation approach. Bowman's group have developed a very efficient general approach to compute 
rovibrational intensity of large polyatomic  MULTIMODE.\cite{MULTIMODE}
Finally we note that Bernath's group
have produced a number of diatomic line lists based on the use of {\sc
level} with the effects of electron spin treated using perturbation
theory, e.g. \citet{14BrBeWe.NH,14MaPlVa.CH,14RaBrWe.CP}.  We note that
there are also a number of studies which are the product of collaboration between the various groups
\cite{jt583,jt635,14BrRsWe.CN,jtNO}.
\red{Shell we mention Perevalos's works on line lists of CO2, N2O, C2H2? Although this route based on effective-rotational Hamiltonians can become endless.}


\section{Method}

The general methodology used by us for constructing line lists has been
extensively discussed elsewhere; in particular Lodi and Tennyson \cite{jt475}
gave an introduction on how to perform such calculations and Tennyson
\cite{jt511} reviewed the methodology used by the ExoMol project which
is summarised in
Fig.~\ref{f:workflow}.

The first step in the calculation is construction of a potential
energy surface (PES) using a high-level, {\it ab initio} procedure,
which we generally do with {\sc MOLPRO} \cite{12WeKnKn.methods}. At
the same time the computation of the appropriate dipole moment
surfaces (DMS) is performed. These then form the input to the
appropriate nuclear motion program; these programs are the focus of
the present article. In only a very few cases
\cite{jt298,jt347,jt506,jtH3+} are completely {\it ab initio}
procedures the best choice for obtaining an accurate line list.  In
general this is only true for systems with very few electrons.
Otherwise it is necessary to refine the calculation using experimental
data.

There are three methods of improving the calculated line list on
the basis of empirical data. The most common one is to refine the PES
in the manner illustrated in Fig.~\ref{f:workflow}. This
methodology, which is widely used by a number of groups
\cite{ps97,01TyTaSc.H2S,88Jexxxx.method}
\red{Sergey: are you aware
of a review which discusses this process? No, sy: I don't know a review. can we cite exomol0 instead?} 
involves either adjusting
parameters in the original fit of the PES or adding an auxiliary function
which captures changes to this PES. The second method, which can only
be used in programs such as {\sc TROVE} which uses uncoupled vibrational
and  rotational basis functions, the so-called $J=0$ representation
of rotational excitation \citet{09YuBaYa.NH3}, involves band origin shifts.
In this method, the vibrational band origins that are computed in
the rotationless ($J=0$) step of the calculation  can be shifted
to the observed one prior to solving the fully-couple rotation-vibration
problem \cite{syref2}. The third method involves substituting
empirical energy levels at the end of the calculation. The format used
for storing ExoMol line lists \cite{jt548,jt631} involves creating
a states file which stores all energy levels and associated quantum
numbers. There are now well-established procedures for extracting
experimentally-determined energy levels from high resolution spectra
\cite{jt412,12FuCsi.method,jt562} and these energies can simply
be used to replace the computed ones in the states file.

The situation with DMS is very different. There evidence is that
these can be calculated {\it ab initio} more accurately than they
can be obtained by inverted experimental data \cite{jt156}. Furthermore
theoretical procedures have been developed which allow the assignment
of uncertainties to individual
transition intensities \cite{jt522,jt625}, although at present these
are too onerous to be used routinely for the very large line lists being
considered here. Reviews discussing the theoretical determination
of accurate DMS have been given by each of us \cite{13Yuxxxx.method,jt573}.

The nuclear motion programs which are the focus of this software review can
be thought of as solving the Schr\"odinger equation implied
by the nuclear-motion Hamiltonian:
\begin{equation}
\hat{H} = \sum_I \frac{-\hbar^2}{2 M_I}\nabla_I^2 + V(\underline{Q})
\end{equation}
where $I$ runs over the nuclei, each of mass $M_I$ and $V(\underline{Q})$
is the (electronic) PES expressed in internal coordinates $Q$.
Of course this expression already assumes the Born-Oppenheimer approximation
and neglects any couplings between PESs. To make progress with
Hamiltonian (1) it is necessary separate out the centre-of-mass coordinates
which represent the translation motion of the whole molecule. The
methods below also all work in body-fixed coordinates which involve
separating the rotations from the vibrations by fixing an axis system
to the molecular frame and using internal coordinates to express the
vibrational coordinates. Precisely how this is done varies between
the different programs.

In the standard variational approach the energies and wave functions
of the body-fixed Hamiltonian are obtained by using appropriate basis
functions to represent the rotational and vibrational motion, and then
diagonalising the resulting matrix (see, for example, review by \cite{08BoCaMe.methods}) For the rotational motions the
choice of basis functions is straightforward as symmetric top eigenfunctions
(Wigner $D$-matrices) form a complete set. For the vibrational
coordinates it is often preferable to use a grid-based discrete
variable representation (DVR) \cite{lc00} rather than actual
functions. Furthermore, rather than diagonalizing the resulting
Hamiltonian matrix in a single step, our approach often uses
intermediate diagonalizing steps so that the final matrix diagonalization
is as compact as possible,

Ones ability to diagonalize the large matrices necessary for obtaining
the many energies and wave functions required for computing hot line lists
usually provides the computational bottleneck in these calculations.
However, the very large number of transition probabilities, which we
generally choose to represent as Einstein A coefficients, that need to be
computed usually means that this step can come to dominate the
actual computer
time used. Measures to mitigate this are discussed below.


\section{Diatomics systems}

Le Roy's program level \cite{lr07} is our choice for computing the spectra
of closed shell ($^1\Sigma$) diatomic molecules. The program has
been refined over many years by Le Roy and has been used by us without
further changes.

However, for more complicated diatomic systems, in particular ones
involving coupled electronic states or non-$\Sigma$ states we have
developed our own program, {\sc Duo}. A first release of {\sc Duo} has
just been published \cite{jt609} and the reader is refereed to this
paper and an associated topical review \cite{jt632} for full details.
{\sc Duo} is still under regular development and number of
improvements to the functionality of the published version
have been made of which we
highlight three here.

First, the published version of {\sc Duo} only considers truly bound
states. However, there are a number of situations where it is necessary
to consider quasi-bound or resonance states, or indeed the continuum
itself. Besides the standard case where rotational excitation leads
to quasi-bound states being trapped by the centrifugal barrier, there
is also pre-dissociation caused by coupling to dissociative states
and spin-orbit effects caused by coupling to such states.
A facility has been added to {\sc Duo} which allows an artificial wall to
placed at large internuclear separation; this has the effect of
discretizing the continuum and allowing localized, resonance states
to be identified \cite{fraser}.

Second, the energy levels of open shell molecules are sensitive to the
effects of magnetic fields. The behaviour of molecules in a magnetic
field provides a spectroscopic tool as well as being important in fields
as diverse as molecular trapping \cite{14BaMcNo.diatom} and astrophysics \cite{02BeSoxx.diatom}.
For an open shell diatomic, the splitting, $\Delta E_{JM}$, due to weak magnetic field of strength
$B$, otherwise known
as the Zeeman splitting, is given by
\begin{equation}
 \Delta E_{JM} = g_J M \mu_0 B
\end{equation}
where $J$ is the total angular momentum quantum number and $M$ is its projection along the direction
of the magnetic field. Here $g_J$ is the effective Land\'e g-factor for the given level.
In the Hund's case (a) basis used by {\sc Duo}, $g_J$ can be evaluated as
\begin{equation}
 g_J = \sum_n |C^{J,\tau}_{\lambda,n}|^2  \frac{\Lambda_n + 2\Sigma_n}{J(J+1)} \label{eq:gJ}
\end{equation}
where $C^{J,\tau}_{\lambda,n}$ is the eigenvector of the $\lambda^{\rm th}$ state with
good quantum numbers $J$ and $\tau$ (parity) and $n$ is a compound basis index
\begin{equation}
 |n> = |{\rm state},J,\Omega,\Lambda,S,\Sigma,v>.
\end{equation}
In this state denotes an electronic state with electron spin $S$ and
$v$ an associated vibrational state. $\Omega (=\Lambda +\Sigma)$,
$\Lambda$ and $\Sigma$ are projection of $J$, the orbital angular
momentum and $S$ onto the body-fixed molecular axis, respectively. In
eq.~(\ref{eq:gJ}), $\Lambda$ and $\Sigma$ are subscripted by $n$ to
emphasize that they are not conserved quantities but their value
depends on the state part of the basis. An extension to {\sc Duo} to
evaluate $g_J$ has recently been written and tested for a few diatomic
systems, notably CrH, C$_2$ and AlO \cite{jtgfac}.

Finally, visualization of wave functions can be very helpful for interpreting results. The latest
version of {\sc Duo} has incorporated plotting routines to aid the inspection of the results.



\section{Triatomic systems}

The {\sc DVR3D} program suite obtains variationally exact solutions
for the bound-state, three-atom nuclear motion problem for a given PES
within the Born-Oppenheimer approximation. The program has been developed
over a number of years originally starting as a finite basis set
procedure \cite{jt20,jt48,jt79,jt128} before evolving \cite{jt129} to
one which is based on the use of DVR in all vibrational coordinates
\cite{jt130,jt160}.  DVR3D and its predecessors have been benchmarked
against other, similar, nuclear motion codes such {\sc VTET},
and indeed {\sc TROVE},
to confirm the accuracy
of both the computed vibration-rotation energy levels
\cite{jt309,jt635} and transition moments \cite{jt78}.

The current published release of the {\sc DVR3D} program suite
\cite{DVR3D} is actually the third but dates back to 2003. Since then
{\sc DVR3D} has undergone a large series of developments, not least to
facilitate the calculation of huge line lists.  All modules have also
been subject to a re-write to both make them more consistent and to
bring the programming up to a more modern programming standard.

Figure~\ref{f:dvr3d} gives the flow structure for the current version
of {\sc DVR3D}. The main driving module, {\sc DVR3DRJZ}, solves the
vibration-only or Coriolis-decoupled vibration-rotation problem.  For
rotationally-excited molecules, the results of {\sc DVR3DRJZ} provide
the basis functions used in one of the {\sc ROTLEV} modules to solve
the fully-coupled vibration-rotation problem; where the choice of
module depends on the axis embedding used. The solutions to
vibration-rotation problem can be used to compute expectation values
of given variable in {\sc XPECT}, this module is particularly used in
performing fits of the PES to spectroscopic data where the
Hellmann-Feynman theorem can be used to evaluate the expectation
values of derivatives of the PES with respect to parameters of the
fit. These same vibration-rotation eigenvectors can be used to compute
line strengths in {\sc DIPOLE} which in turn provides the necessary
information to generate spectra. The new modules and significantly
amended modules have been highlighted in this figure and are
discussed below.

A new module, {\sc RES3D}, has been written \cite{jt443} which can be
used to characterize quasibound or resonance states lying above
dissociation.  The automated procedure for doing the analysis has been
successfully used to study resonances in both H$_3^+$ \cite{jt443} and
water \cite{jt500}. Details of how this module are given below.

In addition the
functionality of {\sc DVR3D} has been increased by a thorough re-write of
the codes in which the $z$-axis is placed perpendicular to plane of the
molecule,the $z$-perpendicular embedding option
\cite{jt290,jt310}.  A new module, {\sc DIPOLE3Z},
is introduced which computes transition dipoles for the
$z$-perpendicular embedding case.


Algorithmic improvements include the following:
\begin{itemize}
\item The automated Gauss-(associated) Legendre quadrature generation
  procedure, which was adapted from one given by Stroud and Secrest
  \cite{jt66} has been replaced by a brute force one which involves
  finding zeros in the polynomial equation \(P_N^k(x)=0\) for $N$
  point quadrature.  This was found to be essential for grids with $N
  > 90$ and has been successfully used for $N$ up to 150 \cite{jt635}.
  The automatic check on the validity of grid obtained by comparing
  summed weights with the analytic value given by Stroud and Secrest
  has been retained.
\item For large calculations, module {\sc rotlev3b}
in  the published version of {\sc DVR3DD} can
spend a long time constructing the final Hamiltonian matrix.
{\sc ROTLEV3b} uses vibrational functions
generated in the first step of the calculation \citep{jt46} to provide
basis functions for the full ro-vibrational calculation performed by
{\sc ROTLEV3b}. For high $J$ calculations this algorithm involves
transforming large numbers of off-diagonal matrix elements to the
vibrational basis set representation, see Eq.~(31) in \citet{jt114}.
This step has been re-programmed as two successive summations rather
than a double summation at the cost of requiring an extra, intermediate
matrix \cite{jth2s}. This had the effect of reducing the cost of Hamiltonian
construction to below that of Hamiltonian diagonalisation, which is
generally the case for the other modules of {\sc DVR3D}.
\item {\sc Dipole3}  by default computes all
  transition dipoles between the bra and ket wave functions it is asked
  to process. For large line lists, computing transition dipoles
  actually dominates computer usage and this can be inefficient.
  These line lists are usually characterized by a lower energy cut-off,
  which determines the temperature range for which the line list is
  valid, and an upper energy cut-off which determines the frequency
  range. Computing transition moments between these ranges is
  expensive and unnecessary. New input variables have been introduced
  to avoid this \cite{jt538}.
\item {\sc DVR3DRJZ} employs an algorithm which relies on solving a Coriolis-decoupled
  vibrational problem for each $(J,k)$, where $k$ is the projection of
  the rotational motion quantum number, $J$, onto the chosen
  body-fixed $z$ axis. This provides a basis set from which functions
  used to solve the fully-coupled ro-vibrational problem are selected
  on energy grounds \cite{jt66}.  Hot line list can involve
  calculations with high $J$ and experience has shown that in this
  case not all $(J,k)$ combinations are actually needed.  An option
  has been implemented where unneeded high $k$ calculations are not
  performed \cite{jt635}. In practice, this does not save much
  computer time, since the initial $(J,k)$ calculations are quick, but
  does save disk space.
\item Again for large calculations, the
algorithm used by module {\sc DIPOLE3} to read in the wave functions
required a lot of redundant reads. {\sc DIPOLE3} has been re-structured to
reduce the number of times the wave functions need to be read \cite{jt635}.
\end{itemize}


Finally, matrix
diagonalisation is the rate-limiting step in most applications
of {\sc DVR3D}. A number of new real, symmetric matrix diagonalisers
have been added to the  LAPACK software package \cite{99AnBaBi.method}.
The diagonalisers implemented in {\sc DVR3D} have been
changes where appropriate.


%%%%%%%%%%%%%%%%%%%%%%%%%%%%%%%%%%%%%%%%%%
\subsection{\label{sec:cap_method} Resonance detection}
%%%%%%%%%%%%%%%%%%%%%%%%%%%%%%%%%%%%%%%%%%
%To calculate the resonant states of \htp\ and \dthp\ the CAP method
%was used. It

Resonances can be detected by the behaviour of states lying in the
continuum upon the introduction of a complex absorbing potential
(CAP). To do this
the dissociating system's PES is augmented with a
complex functional form that absorbs the continuum part of the
wave function. This non-Hermitian Hamiltonian produces
$L^2$ wave functions above the dissociation threshold that represent
the resonant states in question \cite{95RiMexx}.


Formally, an imaginary negative potential that acts on the
dissociation coordinate, $R$, is added to the system's Hamiltonian,
$\hat{H}$:
%
\begin{equation} \label{eq:CAP}
\hat{H}'= \hat{H} - i\lambda W(R)\ \ ,
\end{equation}
%
where $\lambda$ is a parameter used to control the CAP's
intensity. The resulting non-Hermitian Hamiltonian, $\hat{H}'$,
defines the energy of the $n^{\rm th}$ resonance, $E_n$, its width,
$\Gamma_n$, and the corresponding $L^2$ wave function, $\Psi_n$,
through the relationship:
%
\begin{equation} \label{eq:resonance}
  \hat{H}'(\lambda)\Psi_n(\lambda)= \left(E_n(\lambda) - i\frac{\Gamma_n(\lambda)}{2}\right)\Psi_n(\lambda)\ \ .
\end{equation}
%
To solve eq.~(\ref{eq:resonance}), $\hat{H}'$ can be projected on a suitable
basis and diagonalized. In the infinite basis set limit, the
eigenvalues corresponding to the resonant states will be found in the
limit where $\lambda \rightarrow 0$. Fortunately the use of a finite
basis set is both necessary and beneficial: the error introduced by
the CAP and the finite basis set have opposite phase. This implies
that these errors will cancel each other out at some optimal value,
$\lambda_{op}$, thus yielding the complex ``observable'' associated
with the resonant state.

A search for $\lambda_{op}$ is made by studying the behaviour of the
complex eigenvalues of eq.~(\ref{eq:resonance}) with values of
$\lambda$ ranging from zero to a large arbitrary value. This results in
$N$ trajectories in the complex plane, each associated with an
eigenvalue $E_n - i\Gamma_n/2$.  Through graphical analysis of these
trajectories it is possible to identify the point in the complex plane
that corresponds to the optimal value $\lambda_{op}$, and hence
estimate the value for the position, $E_n$, and with, $\Gamma_n$, of
the resonant state. This graphical method consist of locating cusps,
loops and stability points in the eigenvalue trajectory, which are
known to occur in positions around the true eigenvalue for the
resonances on the complex plane.\cite{jt443,81MoFrCe}.

The approach taken in the new {\sc RES3D} module of  {\sc DVR3D} is to first diagonalize $\hat{H}$ of
the system under study and store the basis elements $\phi_i$, and
eigenvalues $\varepsilon_i$ lying near dissociation.  As one
can expand the functions $\Psi_n$ of eq.~(\ref{eq:resonance}) onto the basis
set obtained from the bound state calculation:
%
\begin{equation} \label{eq:wave}
|\Psi_n(\lambda)> = \sum_i c^i_{n}(\lambda)|\phi_i>\ .
\end{equation}
%
The coefficients $c^i_n(\lambda)$, the resonance energies
$E_n(\lambda)$ and the resonance widths $\Gamma_n(\lambda)/2$ can then
be obtained by diagonalizing the Hamiltonian:
%
\begin{equation} \label{eq:capdiag}
H'_{ji}= <\phi_j|\hat{H}'|\phi_i>= \varepsilon_i\delta_{ji} - \imath \lambda
<\phi_j|W(R)|\phi_i> \ ,
\end{equation}
%
where $\varepsilon_i$ is the $i^{th}$ eigenvalue and $\phi_i$ is the
$i^{th}$ eigenvector obtained from the diagonalization of $\hat{H}$.
For systems with many bound states, wave functions associated with the
strongly bound states are not needed when diagonalizing the Hamiltonian so can
be dropped. This means that the  Hamiltonian matrix $H'$ is small and
cheap to diagonalize which is important as the graphical method relies
on many diagonalizations with different values if $\lambda$.

The $H'$ matrix is complex symmetric matrix which therefore yields
complex the eigenvalues needed to characterize both the position and
width of the resonance. {\sc RES3D} uses LAPACK \cite{99AnBaBi.method} routine {\sc zgeev}
to perform this diagonalization.


\section{Polyatomic systems}

\trove\ is a variational method and a Fortran~2003 program to
construct and solve the ro-vibrational \schr\ for a general polyatomic
molecule of arbitrary structure \citet{TROVE}. The kinetic energy
operator is constructed as an expansion in terms of internal
vibrational coordinates with the expansion coefficients obtained
numerically on-the-fly.  The obtained energies and eigenfunctions can
be used to model absorption/emission intensities (absorption
coefficients) for a given temperature as well as temperature
independent line strength and Einstein coefficients
\cite{05YuThCa.method}. The latter is then used as the input to
construct molecular ExoMol line lists. \trove\ provides an integrated
facility for refining the \ai\ PES in the appropriate analytical
representation \cite{11YuBaTe.NH3}. Being a general program \trove\
requires modules for each molecular type with all individual
specifications including a descriptions of the molecular structure,
internal coordinates, and their symmetry properties. \trove\ uses a
symmetry adapted product-type basis set representation with an
automatic symmetrization procedure \cite{16YuYaxx.methods}. The
Hamiltonian matrix constructed by \trove\ is factorized into symmetry
blocks corresponding to different irreducible representations. The
molecular symmetry group \cite{04BuJexx.method} is used to classify
the symmetries.  The construction of the ro-vibrational basis set is
done employing three steps: (i) the 1D basis set functions are
obtained either as numerical solution of 1D \schr\ equations using the
Numerov-Cooley method \cite{23Nuxxxx.method,61Coxxxx.method} or the
Harmonic oscillator wavefunctions; (ii) \schr\ equations are solved
for reduced Hamiltonians for different types of degrees of freedom
connected by symmetry transformations in order to obtain a more
compact, contracted basis set; the eigenfunctions of the $J=0$ \schr\
equation are then contracted and used to form the final ro-vibrational
basis in the $J=0$-representation \citet{09YuBaYa.NH3}. (iii) the
final step involves constructing and diagonalising the symmetrized
ro-vibational.

The program \trove\ has being not only to compute energies and spectra for a series of polyatomic molecules, it has also been used to study some of their properties, for example the so-called rotational energy clustering\cite{09YuOvTh.SbH3,14UnYuTe.SO3} or the temperature-averaged nuclear spin-spin matrix elements\cite{10YaYuPa.NH3}.

Even prior to the ExoMol project a number of modifications had been
implemented to \trove\ subsequent to its original publication \cite{TROVE}.
These have proved to be important for the project.
\begin{enumerate}
  \item Symmetry adapted basis set and contraction scheme based on the reduced Hamiltonian problems  (to be finally reported soon \cite{16YuYaxx.method});
  \item Intensity calculations \cite{09YuBaYa.NH3};
  \item $J=0$ representation \cite{09YuBaYa.NH3};
  \item Thermal averaging using an expansion of the matrix exponent \cite{10YaYuPa.NH3};
  \item Empirical band center corrections \cite{09YuBaYa.NH3}.
\end{enumerate}


 The typical TROVE intensity project consists of the following steps:

\begin{enumerate}
  \item Expansion of the Hamiltonian operator (kinetic and potential energy expansion coefficients on-the-fly) as well as of any `external' function (e.g. dipole moment, polarizability, spin-spin coupling or any other property; PES correction $\Delta V$ used in the refinement process \cite{11YuBaTe.NH3});
  \item Numerov-Cooley solution of the 1D \schr\ equations;
  \item Eigen-solutions of the reduced Hamiltonian problems; \label{i:reduced}
  \item Symmetrization of the contracted eigenfunctions from step~\ref{i:reduced} and construction of the symmetry-adapted vibrational basis set $\phi_{i}$;
  \item Calculation of the vibrational matrix elements for all vibrational-dependent terms in the Hamiltonian operator expansion as well as in the expansion of external functions (e.g. dipole) when required; \label{i:vib}
  \item Diagonalizaitons of the $J=0$ Hamiltonian matrices for each irreducible representations in question;
  \item Conversion of the primitive basis set representation (vibrational matrix elements from step~\ref{i:vib}) to the $J=0$ representation;\label{i:conv}
  \item Construction of the symmetry-adapted rovibrational basis set as a direct product of the $J=0$ eigenfunctions and rigid rotor wavefunctions;
  \item Construction of the rovibrational Hamiltonian matrices for each $J\le 0$ and symmetry $\Gamma$ required;
  \item Diagonalization of the Hamiltonian matrices with eigenvectors stored for the postprocessing (e.g. intensity calculations) if necessary; \label{i:diag-rovib}
  \item For the intensity calculations (line list production), all pairs of the rovibrational eigenvectors (`upper' and `lower') from step~\ref{i:diag-rovib} (subject to the selection rules as well as to the energy, frequency and $J$ thresholds)  are cross-correlated with the dipole moment $XYZ$ components in the laboratory-fixed frame via a vector-matrix-vector product, where the body-fixed $xyz$ components of the dipole moment from step~\ref{i:vib} are transformed to the $XYZ$-frame using the Wigner-matrices. \label{i:int}
\end{enumerate}



The ExoMol project high-temperature applications has drawn new
challenges for \trove\ with demands on very high rotational and
vibrational excitations. This in turn required larger basis sets and
therefore larger Hamiltonian matrices, which increased the costs of
the memory (both for RAM and storage) and the calculation time. For
example, for the SO$_3$ line list \cite{jtSO3} with extremely high
rotational excitations (up to $J=130$) due to the very heavy character
of the molecule, the size of the Hamiltonian matrices to be solved had
to be as large as 400,000$\times$400,000, which represents our biggest
calculations so far. This is despite the fact that only the smallest
($A_1\p$ and $A_1\pp$ symmetries of \Dh{3}) had to be considered due
to the nuclear spin statistics of $^{32}$S$^{16}$O$_3$. The sheer size
of these matrices required special measures not only on the software
side (\trove), which is discussed below, but also from the hardware.
For this example of the 400K$\times$400K matrices, the
diagonalizations were performed on the Cambridge SMP facilities within
the DiRAC~II project, using about 1000 cores, 6 Tb of RAM and a
specially adapted for \trove\ a version of the eigensolver PLASMA by
the SGI team.

In order to tackle these challenges associated with the large basis set and Hamiltonian matrix sizes, the following critical modifications of \trove\ have been performed.

\textbf{Checkpointing.} The complete production of a line list for a
polyatomic molecule with four and more atoms, take long times, longer
than the wall-clock limits of the high performance computers (HPC)
we have access to would allow. Therefore it was
critical to implement `checkpointing' for all calculation steps of the
\trove\ calculation protocol, together with the checkpointing control
mechanism allowing restart at any time. Moreover, the
eigen-coefficients in different representations used at different
stages are also stored as `checkpoints' thus allowing to treat the on
the same footing. In order to prevent accidental usage of the wrong
checkpoints, most of these files contain a built-in structure of
`signatures' with a header containing some key parameters representing
a \trove\ project (expansion orders of the kinetic and potential
energy functions, sizes and types of the basis sets etc) and a
control-phrase in the footer (e.g.
\verb!End Quantum numbers and energies!) as well as at the end of
different sections.


\textbf{Symmetries} Each molecule type in \trove\ is represented as a project defined by its reference (equilibrium) geometry, definition of the geometrically defined coordinate (GDC) and associated symmetry transformation properties of these coordinates and rigid-rotor wavefunctions. The same molecule type allows different coordinates choices of GDC depending on the specific of the system, as different independent functions of these coordinates. Different choices for the reference geometry are also possible. For example as a rigid equilibrium structure or as a non-rigid reference configuration. The transformation symmetry properties of the rigid-rotor wavefunctions $\ket{J,K,\tau}$ \cite{05YuCaJe.NH3} will vary depending on the choice of the $z$-axis. For example in case of an XY\2\ molecule, this can be along the bisecting vector or perpendicular to it, which changes the symmetry properties of $\ket{J,K,\tau}$. For most of the symmetries, the irreducible symmetry adapted combinations of the rigid-rotor wavefunctions are obtained as Wang functions
\begin{eqnarray}
\label{eq:J,K,gamma:1}
\ket{J,K,\Gamma} &=& \frac{1}{\sqrt{2}} \left[ \ket{J,K} \pm \ket{J,-K}  \right],  \quad (K\ne 0) \\
\label{eq:J,K,gamma:2}
\ket{J,0,\Gamma} &=& \ket{J,0}.
\end{eqnarray}
Because of this property and that the Hamiltonnian operator is quadratic in terms of the angular  momentum operators $\hat{J}_x$, $\hat{J}_y$,  and $\hat{J}_z$, the ro-vibrational Hamiltonian matrix has a block-diagonal structure with zero matrix elements for $|K-K'|>2$.

In the course of the ExoMol project the following new molecular types and corresponding symmetries were implemented:   XY\4\ (\Td, \Cv{3}) \cite{13YuTeBa.CH4,14YuTeBa.CH4}, non-linear and non-rigid  X\2Y\2\ (\Cs, \Dh{2}, \Ch{2}$^{+}$, \Ch{2}, \Cv{2}) \cite{13PoKoMa.H2O2,15AlOvPo.H2O2}, linear X\2Y\2\ (\Dh{n}, \Cv{2}, \Cs), rigid X\2Y\4\ (\Dh{2}).

\textbf{Euler symmetry}
The XY\4\ is a special case since the simple
symmetrization rules in
Eq.~(\ref{eq:J,K,gamma:1},\ref{eq:J,K,gamma:2}) do not work
\cite{11AlLeCa.CH4} due to the additional symmetry axis (1,1,1)
required to define the equivalent rotations~\cite{99BuJexx.CH4}. The
adapted basis set is given in this case by a linear combination of all
$\ket{J,k}$ ($-J \ge k \ge J$). To address this issue a special
symmetrization routine for implemented. As a consequence the
rovibrational Hamiltonian matrix does not have the block-diagonal
structure, and therefore the corresponding module with matrix elements
had to be modified. The same applies to the matrix elements of the
dipole moment components in the laboratory-frame for the \Td:
abandoning Eq.~(\ref{eq:J,K,gamma:1},\ref{eq:J,K,gamma:2}) leads to a
less compact $K$-structure and has to be treated differently. More
details can be found in our paper on presenting our methane line
list\cite{jt564}.




\textbf{`Cluster' of the poyad number}


\subsection{PES refinement}


The \trove-refinement method \cite{11YuBaTe.NH3} is based on the two main features: (i) the eigenfunctions of the rovibrational Hamiltonians (usually $J\le 5$) corresponding to the \ai\ PES $V^{\rm ai}$ are used as basis functions to solve the \schr\ equations for the modified PES  $V^{\rm R}$ during the refinement procedure, which (ii) is represented as given by
\begin{equation}
V^{\rm R} = V^{\rm ai} + \Delta V.
\end{equation}
The refined part of the PES $V^{\rm R}$ is in turn represented as an expansion in terms of the internal coordinates
\begin{equation}
\Delta V = \sum_{ijk\ldots} \Delta F_{ijk\ldots } \xi_1^i \xi_2^j \xi_3^k \cdots .
\end{equation}
with the expansion coefficients $\Delta F_{ijk\ldots }$
being varied using the Helmman-Feynmann theorem. This term $\Delta V$ plays the role of the external function at step~\ref{i:vib}.  The refinement \trove\ project requires the following additional calculation steps starting from step~\ref{i:int} in the intensity calculation protocol
\begin{enumerate}
 \setcounter{enumi}{1000}
  \item The vibrational matrix elements of  $\Delta V$ (for a given approximation) are converted from the $J=0$ representation (step~\ref{i:conv}) to the representation of the \ai\ rovibrational eigenvectors;
  \item At each iteration, a set of corrected with $\Delta V$ Hamiltonian matrices are constructed and diagonalized;
  \item The Eigenvalues are compared to the experimental values;
  \item The diagonal matrix elements of $\xi_1^i \xi_2^j \xi_3^k$ on these eigenfunctions are computed and used to evaluate the next approximation for $\Delta V$;
  \item The iteration steps are repeated until all criteria are satisfied.
  \red{CONSTRAINT TO AB INITIO?}

\end{enumerate}


\subsection{Curvilinear coordinates}

Originally TROVE has been based on the expansion in terms of
linearized coordinates around an equilibrium geometry in case of rigid
molecules or a 1D non-rigid reference configuration
\cite{83Jensen.method} in case of molecules with large amplitude
molecules (e.g. ammonia or hydrogen peroxide). Linearized coordinates
are defined as a linear expansion of GDCs in terms of the Cartesian
coordinates displacements truncated after the linear term (see, for
example, \cite{98BuJexx.method}). The linearized coordinates have the
advantage of simplifying the Eckart conditions \cite{TROVE}.  Very
recently this approach in TROVE has been extended for expansions in
terms of geometrically defined (or curvilinear) coordinates. In order
to be able to use the Eckart conditions in this case, an automatic
differentiation (AD) procedure has been implemented (see
Ref.~\cite{15YaYuxx.method} for details). Use of curvilinear coordinates
significantly improves the basis set convergence\cite{15YaYuxx.method}.
This method was originally tested on
NH\3, PH\3\, CH\3Cl and H\2CO, and has been used in subsequent
applications \cite{15OwYuYa.SiH4,15OwYuYa.CH3Cl,jt634}. AD is a robust
numerical method to compute derivatives of arbitrary functions by
computer programs.



\subsection{TROVE for a linear molecule}

TROVE has been originally written to treat non-linear molecules only.
This meant that TROVE was not capable of treating accurately enough
for practical spectroscopic applications even molecules such as water,
which has a relatively low barrier to the linearity.  This has been
addressed in the most recent version of TROVE by extending it to the
so-called $3\times N-5$ approach, where the rotation of the molecule
around the molecular axis ($z$) is excluded from the set of the Euler
angles and combined with the vibrational set of modes. Technically
this is done by describing the deformation (angle from the linearity
$\alpha$) and rotation of the molecule via one double degenerate
coordinate ($q_x$,$q_y$) as projections of $\alpha$ onto the planes
$xz$ and $yz$. The linearized coordinates in this cases are best
suited for this 2D internal mode. All kinetic energy terms
corresponding to the $z$ component are simply set to zero and thus
excluded from the calculations. Thus the construction of the $3\times
N-5$ Hamiltonian requires minimal modifications of the $3\times N-6$
code. However there ro-vibrational basis set in the product form
$\ket{J,k}\ket{v,l}$ has to be constrained as follows
\cite{70Watson.method}
$$
  k \equiv l = \sum_i l_i
$$
where $k$ is the rotational quantum number (projection of the rotational angular momentum on $z$) and $l_i$ are the vibrational angular momentum and $v$ is a generic vibrational quantum number. Thus the vibrational basis set has to be constructed with $l$ as a `good' quantum number. To this end the $N$D isotropic Harmonic oscillators are used as basis functions. The symmetrization procedure had to be modified to allow for the $l$-classification of the basis set. The $3\times N -5 $ approach will be reported elsewhere.


\subsection{GAIN}

The longest part of the line list production is the intensity calculations. The hot line lists of polyatomic molecules typically required billions transitions (linestrengths or Einstein coefficients) to be computed. Each calculation is a vector-matrix-vector product, each of which is relatively small in terms of the memory costs is fully independent from other transitions. This makes it perfectly suitable for the GPU architecture. We have modified the intensity part of TROVE to make it compatible for and efficient with GPUs. The new TROVE module  and the underlying approach is called GAIN \cite{jtGAIN}, which in fact with small modifications could be adopted for other variational programs. The gain in the calculation speed is from 10 to 1000 depending on the type of GPU used.


\section{Larger molecules}



As part of the ExoMol project we have worked with one further nuclear motion code
{\sc AngMol} which was originally developed by Gribov and Pavlyuchko \cite{88GrPa.method}.
With Pavlyuchko we developed a hybrid variational -- perturbation theoretical method for
treating both vibrational and vibrational-rotational motion \cite{jt588} and  computing
spectra of large, hot systems efficiently \cite{jt603}. This methodology has been used
successfully to obtain a  line list for hot nitric acid (HNO$_3$) \cite{jt614}.
However, {\sc AngMol} has been developed in a highly specific manner. Rather than continuing
its development, our plan is to implement the hybrid procedure successfully tested in
{\sc AngMol} within {\sc TROVE}.


\section{Conclusions and future developments}

The codes {\sc Duo}, {\sc DVR3D} and {\sc TROVE} are all publicly
accessible via the CCPForge program depository
(https://ccpforge.cse.rl.ac.uk/), where each of them are available as
a separate project.

A number of future developments of these codes are in progress or
being planned. In particular we are just starting to extend the polyatomic
codes include transitions between different electronic states and hence
to consider the vibronic transitions already considered by the diatomic
code {\sc Duo}. The calculation of all states up to dissociation,
something that has been possible with  {\sc DVR3D} for some time
\cite{jt100,jt132,jt230}, leads to the possibility that these wave functions
can be used for reactive problems just above dissociation. This possibility
is currently being explored \cite{jtfaraday}.

Other?





\subsection*{ACKNOWLEDGMENTS}
This work was supported by the ERC under Advanced Investigator Project 267219.
We thank the other members of the ExoMol team for their participation
in the many program developments discussed in this article.
\bibliographystyle{apsrev}
\bibliography{journals_phys,jtj,exogen,methods,additional,linelists,diatomic,NH,CH,CP,CN,H2S,programs,books,CH4,NH3,SO3,H2O2,SbH3,SiH4,CH3Cl}


\clearpage
%%%%%%%%%%%%%%%%%%%%%%%%%%%%%%%%%%%%%%%%%%%%%%%%%%%%%%%%%%%%%%%%%%%%%%%%%%%%%%%%%
% FIGURE CAPTIONS

%%%%% FIGURE ---- qc.eps
\begin{figure}
\includegraphics[height=100mm]{workflow.eps}
\caption{\label{f:workflow} A typical work flow of the line list production.  }
\end{figure}

\begin{figure}
\centering
\includegraphics[height=100mm]{fig2.eps}
\caption{\label{f:dvr3d} Flow chart illustrating the various modules of program
{\sc DVR3D}. Molecules in bold are new and those in italic have significant
algorithmic improvements since the last published release of the code \cite{DVR3D}.}
\end{figure}


\begin{figure}
\centering
\includegraphics[height=100mm]{trove.eps}
\caption{\label{f:trove} Flow chart illustrating different stages of the \trove\ algorithm \cite{DVR3D,TROVE,15YaYuxx.method}.}
\end{figure}


\end{document}

